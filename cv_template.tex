\documentclass{article}

\usepackage{titlesec}
\usepackage{titling}
\usepackage[margin=0.5in]{geometry}
\usepackage[hidelinks]{hyperref}
\hypersetup{
    colorlinks=true,
    urlcolor=linkcolor,
    pdfborderstyle={/S/U/W 1}
}
\usepackage[absolute]{textpos} % showboxes
\usepackage{graphicx}
\usepackage{mathptmx}
\usepackage{multicol}
\usepackage[T1]{fontenc}
\usepackage{fontawesome}
\usepackage[framemethod=tikz]{mdframed}

\titleformat{\section}
{\huge}
{}
{.25em}
{\bfseries}

%% Space around title
\titlespacing*{\section}{0pt}{0.3\baselineskip}{0.2\baselineskip}

\definecolor{headercolor}{rgb}{0.122, 0.435, 0.698}
\definecolor{linkcolor}{rgb}{0.192, 0.349, 0.408}

\newmdenv[
]{headerbox}

\newcommand{\ul}[1]{\underline{#1}}

\renewcommand{\maketitle}
{
    % \begin{headerbox}
    \begin{center}
        {\huge\bfseries\theauthor}

        \vspace{.25em}

        \Large{Systems Engineer}

        \vspace{.25em}

        \large{\thetitle}

        \vspace{.25em}

        \faMapMarker: Braga, Portugal |
        \href{mailto:pedro.mendes.26@gmail.com}{\faEnvelope: pedro.mendes.26@gmail.com} |
        \href{https://github.com/mendess}{\faGithub: mendess} |
        \href{https://www.linkedin.com/in/mendes2526/}{\faLinkedinSquare: mendes2526}

    \end{center}
    % \end{headerbox}
}

\setlength{\parindent}{0pt}

\pagenumbering{gobble}

\begin{document}
\title{Curriculum Vitae}
\author{Pedro Mendes}

\maketitle

\hrule

\section{Skills}

\begin{multicols}{2}

    \textbf{Programming:} Rust, C, eBPF, C++, Kotlin, Bash, Typescript, Python,
    Java, C\#, Javascript, Ruby, Haskell, Elixir.

    \textbf{Spoken:} Portuguese Native, English C2.

    \textbf{Markup} \LaTeX, Markdown, Html, CSS\@.

    \textbf{Databases:} SQL (mysql, postgres), NO-SQL (mongodb, neo4j).

    \textbf{Tools:} Advanced Git knowledge, Linux, GNU core utils, Docker.

\end{multicols}

\section{Experience}

\begin{tabular}{p{0.11\linewidth}p{0.73\linewidth}l}

    2019--$\infty$ & \ul{\textbf{Open source contributions}} & \href{https://github.com}{Github} \\
    &
        \href{https://github.com/rust-lang/rust/pulls?q=author\%3Amendess+}
            {rust-lang/rust}
        \hspace{2em}
        \href{https://github.com/rust-lang/impl-trait-utils/pulls?q=author\%3Amendess+}
            {rust-lang/impl-trait-utils}
        \hspace{2em}
        \href{https://github.com/libbpf/libbpf-rs/pulls?q=author\%3Amendess+}
            {libbpf/libbpf-rs}
        \hspace{2em}
        \href{https://github.com/serenity-rs/serenity/pulls?q=author\%3Amendess+}
            {serenity-rs/serenity}



        \href{https://github.com/crate-ci/typos/pulls?q=author\%3Amendess+}
            {crate-ci/typos}
        \hspace{2em}
        \href{https://github.com/waycrate/swhkd/pulls?q=author\%3Amendess+}
            {waycrate/swhkd}
        \hspace{5em}
        \href{https://github.com/bread-graphics/breadx/pulls?q=author\%3Amendess+}
            {bread-graphics/breadx}
    & \\

    2023--Now & \ul{\textbf{Systems Engineer @ Cloudflare}} & \href{https://cloudflare.com}{Cloudflare} \\
    &
        At Cloudflare I work in the emergent technologies team, making use
        of \textbf{eBPF}, \textbf{WASM} and other new technologies to develop prototypes for new
        Cloudflare projects.

        One of which is the \href{https://github.com/cloudflare/daphne}{Daphne}
        project, which implements the
        \href{https://datatracker.ietf.org/doc/draft-ietf-ppm-dap/}{DAP}
        protocol, with the goal of providing Privacy Preserving Measurements. On
        this project I was tasked with scaling the existing prototype, making
        sure it was performant, scalable, observable and correct using Cloudflare's
        \href{https://developers.cloudflare.com/durable-objects/}{Durable Objects}
        as a distributed strongly consistent database.
    & \\

    2021-2022 & \ul{\textbf{Backend Engineer, Library Developer @ Speechify}} & \href{https://speechify.com}{Speechify} \\
    &
        At Speechify I integrated a team dedicated to rewriting the core
        of the experiences in the \textbf{Kotlin Multiplatform} stack, which transpiles to
        ObjectiveC and JS, as well as compiling to jvm bytecode for android,
        which meant understanding how the different memory models and async
        runtimes of these 3 ecosystems interact.
    & \\

    2021-2021 & \ul{\textbf{Full stack and IoT engineer @ Emitu}} & \href{https://emitu.com}{Emitu} \\

\end{tabular}

\section{Highlighted Projects}

\begin{tabular}{p{0.11\linewidth}p{0.73\linewidth}l}
    % PROJECTS
\end{tabular}

\section{Education}

\begin{tabular}{p{0.15\linewidth}p{0.55\linewidth}p{0.26\linewidth}}

    2019 --- Unfinished &
        \ul{\textbf{Masters in Information Systems and Computer Engineering}}
    &
        \textbf{Instituto Superior Técnico}
    \\
    & Forensics Cyber-Security: 16/20 & Average: 15/20 \\
    & Network and Computer Security: 15/20 &\\
    & Advanced Programming: 18/20 &\\

    2015 --- 2019
    &
        \ul{\textbf{Bachelors in Science of Computer Engineering}}
    &
        \textbf{University of Minho}
    \\
    & Informatics Labs (1, 2, 3 and 4) Average: 19/20 & Average: 15/20 \\
    & Program Calculus: 20/20 \hspace{2em} Compilers: 17/20 &\\

\end{tabular}

% \section{Interests}

% Mentoring\@; Systems Programming\@; API Design\@; Automation\@; Linux


\end{document}
